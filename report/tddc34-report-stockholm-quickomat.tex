\documentclass[twocolumn]{article}

% package loading
\usepackage[T1]{fontenc}
\usepackage[utf8]{inputenc}
\usepackage{lmodern}
\usepackage{amsmath}
\usepackage{graphicx}
\usepackage{amsfonts}
\usepackage{hyphenat}

\usepackage{natbib} % Harward style references

\providecommand{\keywords}[1]{\textbf{\textit{Index terms---}} #1}

\title{Report on Quickomat}
\author{Johan Angelstam, Guanqun Li, Renquan Wang, Yunsheng Kong\\
  \{johan791|guali867|renwa331|yunko064\}@student.liu.se}
%\date{2014}

\begin{document}
\twocolumn[
  \begin{@twocolumnfalse}
    \maketitle
    \begin{abstract}
      ...
    \end{abstract}
    \keywords{quickomat, user interface evaluation}
    \vspace{2em}
  \end{@twocolumnfalse}
  ]
\section{Introduction}
This report was as an evaluation project of the course in technical, economical and societal evaluation of IT-products in Linköping University.

\subsection{Background}
In this section, we present some basic background information on vending machines, Quickomat, usability of information systems and safety condition about public transportation. We will introduce the main motivation and goals for this report, as well as the framework and methodology we studied for the evaluation of this area.

\subsubsection{Quickomat}
Quickomat is a vending machine which is operated by the company Quickomat AB. The machine is often found at public places such as train stations, shopping malls and universities in the south of Sweden. Through the machines Quickomat offers public transport authorities and operators a service for ticket sales.

Using the Quickomat service, operators can easily sell their tickets from hundreds of new locations without any investments in infrastructure. Customers can buy tickets for public transportation and concerts or top-up their mobile phone account from the nearest machine. In 2008 the company had around 50 machines in use, \cite{RealDeals2008}.

\subsubsection{Usability of IS}
Definitions of IS usability implies that thus can be seen as “\emph{How well an IS supports its functions}”, in terms of user satisfaction, efficiency and effectiveness. And based on these factors, we can partly evaluate them indirectly through questionaires about attitudes or estimates of ease and frequency of use or what user
interfaces are better for the regular interactive actions between machine and people(Mathias Cöster, Nils-Göran Olve, Åke Walldius, 2012).

\subsection{Motivation \& Goals}
Our motivation to do this research report is based on the concerns as below.

In the first place, Quickomat is quite a popular and public vending machine in Sweden now and closely connected with our normal life everyday. For instance, if you want to take the bus to visit some other places in Linköping now, you have to use the system to buy the bus card which is pre-charged and pay for your trip by it when you get
on a bus. No cash is wanted in the bus transportation nowadays in Linköping.

Secondly or the most important part, by carrying out the researches and discussions about this system, we really want to figure out what this machine or system’s appearance has influenced us in our regular life or made any changes about our living habits, especially in some aspects like public security and interface-usability. Such as the
questions that “\emph{Is it more convenient to use Quickomat compared with the old ways of topping up the phones or buying the tickets?}”, “\emph{Does Quickomat need any improvements  about its interface-usability?}” or “\emph{Has the emergence of Quickomat reduce the criminal events of public transportation?}”.

Finally, it’s a kind of system that we can achieve quite many open resources in public places when we suppose to do the evaluation researches, like the people resources, the machine, the cards and most importantly the company, which is local and it could be more convenient and easier to get in touch with them for further informations compared with other options.

So, based on all the above, the goal of this report is mainly to take a deep look into the Quickomat machine and try to solve the questions we have raised above from the technical and social perspectives and what kind of impacts can be achieved or observed from the evaluation of these perspectives.

\section{Theoretical Framework}
The theoretical framework that we are going to discuss will be the base of our study.

\subsection{Perspectives on Evaluation}

\subsection{Technical Evaluation}
Usability describe the extent that user is enabled to perform specified goals effectively and efficiently (Melody Y. Ivory, 2001), and it is recognized as an important factor in software quality. To evaluate the usability of the information system became necessary in software system. People generally associate usability with some common attributes: learnability, efficiency, memorability, error handling, and user satisfaction (Ben Shneiderman, 1992).

The common approach to evaluate usability is user-based test, which is conduct by a group of people and let them to work on real tasks with the system. Then how easy it is for people to perform each task are collect (Joseph S. Dumas, 1993).

Two others main category of usability evaluation is often used in organizations. Expert-based evaluation are base on analysis of experts. With standards set beforehand, they will check if the system conform to it. An example of these methods that Jakob nielsen mention (1994) is call “\emph{heuristic evaluation}”, which define a lot of guidelines as “\emph{heuristics}”, and let evaluators examine the system according the the heuristics to check the degree that the system compliance to the guideline. Theory-based are most based on models and theories which have more formal procedures to perform the evaluation, which can get more detailed analysis of usability evaluation. One of these methods present by Z Zhang (1996) is the Goal/Question/Metric (GQM) method. In this method, goals are decomposed into tasks. Based on the models, question are designed to cover the goals. Metric are need to check the compliance of the question to the goals. The use scenario is divided to three groups: novice using, error-free expert using and error handling.

\subsection{Social Evaluation}
Social evaluation (SE) is a very important element within an effective evaluation system. One of the most important components of SE is to maintain the safety of work systems in the workplace. Safety of work systems is a function of many factors which affect the system, and these factors affect the safety of work systems simultaneously. For this reason, measuring work system safety needs a holistic approach. In this study, the work safety issue is studied through the analytic hierarchy process (AHP) approach which allows both multi-criteria and simultaneous evaluation. 

The AHP introduced by Saaty (1977) is a method addresses how to determine the relative importance of a set of activities in a multi-criteria decision problem. AHP aims to settle the conflict between practical demand and scientific decision making, and it also make it possible to find a way to blend process qualitative analysis and quantitative analysis (Mu, 1997). Decisions made using the AHP occur in two sequential phases: hierarchy design, which involves decomposing the decision problem into a hierarchy of interrelated decision elements (i.e., goal, and evaluation criteria); and hierarchy evaluation, which involves eliciting weights of the criteria and synthesizing these weights and preferences to determine alternative priorities (Sanjay and Ramachandran, 2006). One of the main advantages of the AHP method is the simple structure. Moreover, the use of AHP does not involve cumbersome mathematics, thus it is easy to understand and it can effectively handle both qualitative and quantitative data (Cengiz et al., 2003).

The AHP is one of the extensively used MCDM methods (Bagranoff, 1989, Arbel and Orgler, 1990 and Moutinho, 1993). It has been successfully used in maintenance policy selection (Arunraj and Maiti, 2010), environmental decision-making (Patrick and Laurence, 1999 and Chiang and Lai, 2002), resources planning (Willett and Sharda, 1991), and conflict management (Saaty, 1990a and Kang et al., 2007). AHP has also been employed for the risk assessment (Mustafa and Al-Bahar, 1991, Gaudenzi and Borghesi, 2006 and Zayed et al., 2008).
Another limitation faced in SE is the inability to measure the variables exactly and objectively. Generally, the factors affecting work system safety have non-physical structures. Therefore, the real problem can be represented in a better way by using fuzzy numbers instead of numbers to evaluate these factors. In this study, a fuzzy AHP approach is proposed to determine the level of faulty behavior risk (FBR) in work systems. In the application, factors causing faulty behavior are weighted with triangular fuzzy numbers in pairwise comparisons. These factors are evaluated based on the work system by using these weights and fuzzy linguistic variables. As a result of this evaluation FBR levels of work systems are determined and different studies are planned for work systems according to the FBR levels. In this way, faulty behavior is prevented before occurrence and work system safety is improved.

\section{Discussion}
\section{Summary}

% Bibliography
\cite{*}
\bibliographystyle{plainnat}
\bibliography{tddc34-report-stockholm-quickomat}
\end{document}