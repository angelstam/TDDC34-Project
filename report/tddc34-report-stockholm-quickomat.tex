\documentclass[twocolumn]{article}

% package loading
\usepackage[T1]{fontenc}
\usepackage[utf8]{inputenc}
\usepackage{lmodern}
\usepackage{amsmath}
\usepackage{graphicx}
\usepackage{amsfonts}
\usepackage{hyphenat}

\usepackage{natbib} % Harward style references

\providecommand{\keywords}[1]{\textbf{\textit{Index terms---}} #1}

\title{Report on Quickomat}
\author{Johan Angelstam, Guanqun Li, Renquan Wang, Yunsheng Kong\\
  \{johan791|guali867|renwa331|yunko064\}@student.liu.se}
%\date{2014}

\begin{document}
\twocolumn[
  \begin{@twocolumnfalse}
    \maketitle
    \begin{abstract}
      ...
    \end{abstract}
    \keywords{quickomat, user interface evaluation}
    \vspace{2em}
  \end{@twocolumnfalse}
  ]
\section{Introduction}
This report was as an evaluation project of the course in technical, economical and societal evaluation of IT-products in Linköping University.

\subsection{Background}
In this section, we present some background information on the Quickomat, usability of IS (Information Systems) and security condition about public transportation, which are the foundations for our further investigation focus and research. We will introduce the main motivations and goals for this report, as well as the framework and methodology we studied for the evaluation of this area.

\subsubsection{Quickomat}
Quickomat is a vending machine which is operated by the company Quickomat AB. The machine is often found at public places such as train stations, shopping malls and universities in the south of Sweden. Through the machines Quickomat offers public transport authorities and operators a service for ticket sales (Quickomat AB, 2012).
    Using the Quickomat service, operators can easily sell their tickets from hundreds of new locations without any investments in infrastructure. Customers can buy tickets for public transportation and concerts or top-up their mobile phone account from the nearest machine. In 2008 the company had already had around 50 machines in use (\cite{RealDeals2008}).  “Our mission is to own and operate a network of ticket vending machines in public places and offer companies the distribution and sale of products through vending machines”, said Staffan Johansson, CEO of Quickomat AB in 2012 (His Svanberg, 2012), and obviously, it has got attentions and made some achievements as he wanted now in the public service area in Sweden.

\subsubsection{Usability of IS}
As the Quickomat is a kind of the special combination of information system and hardware, the interface-usability is a main part we choose to focus on in our report. The usability of IS (Information System) is now regarded as a very important factor for evaluating the quality of software as well as other factors such as  functionality, error handling or reliability. And with the development and growth of the application domain and user population, it gets more and more attentions during every stage of the software or product process such as design, implementation or testing(Zhijun Zhang and Victor R.Basili, 1996). Definitions of IS usability implies that it can be seen as  “\emph{How well an IS supports its functions}”, sometimes in terms of user satisfaction, efficiency and effectiveness. And based on these factors, we can partly evaluate them indirectly through questionaires about users’ attitudes or estimates of ease and frequency of use or what user interfaces are better for the regular interactive actions between machine and human (Mathias Cöster, Nils-Göran Olve, Åke Walldius, 2012).

\subsubsection{Public Transportation Security}
Another aspect we decide to investigate is the influence of the Quickomat’s emergence and development on the public transportation security. As we all know, public transportation security and safety is a widely concerning issue among the citizens and governments. “How to reduce the criminal rates of a city” has become a serious and hot topic in urban areas. Most of time the public transportation planners ignore the security factor when they design a system while considering the other factors or attributes like travel time, cost, convenience and availability.  Also, the occurrence of crime and vandalism on public transportation can be seen as a big problem of violence against the citizens and public order, which may cause a series of other issues like people are afraid of public transportation and change their habits or life-styles with the concern of their personal security (Lester A.Hoel, 2002 ). 

\subsection{Motivation \& Goals}
Our motivation to do this research report is based on the reasons as below.

In the first place, Quickomat is quite a popular and public vending machine in Sweden now and closely connected with our regular life everyday (His Svanberg, 2012). For instance, if you want to take the bus to visit some other places in Linköping now, you have to use the system to buy the bus card which is pre-charged and pay for your trip by it when you get on a bus. No cash is wanted in the bus transportation nowadays in the places of South Sweden, like Linköping.

Secondly or the most important part, by carrying out the researches and discussions about this system, we really want to figure out what this machine or system’s appearance has influenced us in our regular life or made any changes about our living habits, especially in some aspects like public security and usability of IS, as well as answering the questions such as
\begin{enumeration}
  \item “Is it more convenient to use Quickomat compared with the old ways of topping up the phones or buying the tickets?”
  \item “Does Quickomat need any improvements  about its interface-usability?”
  \item “Has the emergence of Quickomat reduce the criminal events of public transportation, like bus robbery or drivers’ safety?”.
\end{enumeration}

Finally, it’s a kind of system that we can achieve quite many open resources in public places when we suppose to do the evaluation researches, like the people resources, the machine, the cards and most importantly the company, which is local and it could be more convenient and easier to get in touch with them for further informations compared with other options(Quickomat AB, 2012). 

Based on all the above motivations, the goal of this report mainly focus on the issue that taking a deep look into the Quickomat, trying to answer the questions we have raised above from both the technical and social perspectives and finding out what kind of impacts can be achieved or observed from the evaluation of these perspectives. And we will mention our opinions on this area from the economic and macro perspectives, too.

\subsection{Limitations}
First of all, we set a limitation for the scope of our evaluation, which focuses on evaluating the technical aspect - interface-usability of Quickomat and social aspect - its influence on public transportation security. 

Secondly, we can be sure that, as the short time schedule of this project and the lack of data and resources about the security issue on the public transportation, we may not be able to complete the full evaluation of security part of Quickomat. But we are really curious and interested in this part and we will try our best to set up the evaluation framework for this and explain how it will work if we have all the resources. 

As for other aspects, like economic and  macro evaluation, we will limit the scale of analysis and give a brief introduction of the them.

\subsection{Framework \& Methodology}
Since we are trying to evaluate the Quickomat machine, we will do some researches and try to use different frameworks and methodologies to make it happen.

Our main methodology of the interface-usability evaluation will be using a popular and famous empirical framework in the field of usability evaluation - the “\emph{Heuristic Methodology}” (Nielsen and Molich, 1990; Nielsen 1994), which, in a short introduction, is a method that involving a set of evaluators examining the system and judging its compliance with recognized usability rules(the heuristics) of all kinds of use aspects (Zhijun Zhang and Victor R.Basili, 1996). 

While, for the security evaluation, we consider it as part of social evaluation and are going to use the AHP (Analytic Hierarchy Process) approach to take a deeper investigation of security issue on the public transportation with the influence of Quickomat. 

During the evaluation, we also implement the questionaire method consisting of some critical incidents questions about the information we want to achieve the data for collection and analysis.

Furthermore, economic and macro evaluation are out of our purpose for the case study and we will just introduce some theoretical frameworks or methodologies from these perspectives.

\section{Theoretical Framework}
This section we will discuss the theoretical framework that we are going to use for our case evaluation. Firstly, we discuss the perspectives on evaluation, and then it follows by the introductions of IT evaluation we investigated for this area.

\subsection{Perspectives on Evaluation}
From the year 2010, the cash was stopped to use on buses and trams in some areas of Sweden based on the information from Östgötatrafiken, which is a company responsible for public transport and the mobility service in Östergötland (BJÖRSÄTER, 2010). And that’s a start of the more and more popular use of Quickomat, we surpose, before the fast expand development of all kinds of other services of Quickomat in the future. 

While, to evaluate this kind of vending machine, we can come up with many ideas and aspects since it’s been regarded as a special product combining the public service and IT (Information Technology). Finally, we choose technical and social aspect to evaluate, because we really want to see how it works in the usability and security for the benefits of public service, and how to improve that on the basis of the knowledge we know.  As below, we will introduce two methodologies we will use in our evaluation and explain how they implement for the Quickomat evaluation. 

In information systems, stakeholders of the system approach the same issue or process differently based on their own perspective. A common problem with is that, if you don’t try to be precise in naming the system, you become confused as to who is performing the activities of the system and what those activities should be. CATWOE framework is often used to produce a Root Definition for each transformation, which names the system in a structured way and making it clear who performs what task, for what purpose.

\begin{description}
  \item[Customers]
    Those who are affected by Transformation, the victim or beneficiary.
  \item[Actors]
    Those who will perform the activities involved in the transformation process.
  \item[Transformation]
    Describe the single process that will convert the input into the output.
  \item[Weltanschauung]
    The view which makes the transformation worthwhile.
  \item[Owners]
    Those who has the authority to make changes happen.
  \item[Environment]
    The constraints/restrictions which may prevent the system from operating.
\end{description}

\subsection{Technical Evaluation}
Usability describe the extent that user is enabled to perform specified goals effectively and efficiently (Melody Y. Ivory, 2001), and it is recognized as an important factor in software quality. To evaluate the usability of the information system became necessary in software system. People generally associate usability with some common attributes: learnability, efficiency, memorability, error handling, and user satisfaction (Ben Shneiderman, 1992). 

The common approach to evaluate usability is user-based test, which is conduct by a group of people and let them to work on real tasks with the system. Then how easy it is for people to perform each task are collect (Joseph S. Dumas, 1993).

Two other main categories of usability evaluation are often used in organizations. Expert-based evaluation is based on analysis of experts. With standards set beforehand, they will check if the system conform to it. An example of these methods is the famous “\emph{Heuristic Evaluation}” (Nielsen, 1994), which define a lot of guidelines as “\emph{heuristics}”, and let evaluators examine the system according the the heuristics to check the degree that the system compliance to the guideline. Theory-based are most based on models and theories which have more formal procedures to perform the evaluation, which can get more detailed analysis of usability evaluation. One of these methods present by Z Zhang (1996) is the Goal/Question/Metric (GQM) method. In this method, goals are decomposed into tasks. Based on the models, question are designed to cover the goals. Metric are need to check the compliance of the question to the goals. The use scenario is divided to three groups: novice using, error-free expert using and error handling.

In our case study, we choose the “\emph{Heuristic Evaluation}” as our method for the reason of that it’s fast and easy to carry out, and is flexible to use. Guidelines are summarized of the design based on the research, which including dimensions, how to engage to use, privacy, the use of colors, icons and how to structure menus. By making user perform specific operation, we collect the results for analysis.

\subsubsection{Evaluation Plan}
Try to answer the question “\emph{How to test the interface?}”  with choosing the approaches like “Develop a set of tasks and ask the evaluators to carry them out” or like “Provide evaluators with the goals of the system and allow them to develop their own tasks” (Nicky Danino, 2001).

\subsubsection{Choose Evaluators}
This is an important step. The more evaluators we set and choose in the process, the more usability problems we may find in the end. And also the selection of evaluators is also related with the result, which can be divided into two categories normally, those with experience and without experience (Nicky Danimo, 2001). We can also see the impact of numbers of evaluators on the proportion of usability from Figure 1, which emphasize its important status in the heurisitic evaluation.

% Figure 1 (Jakob Nielson,1995)
% Curve showing the proportion of usability problems in an interface found by heuristic evaluation using various numbers of evaluators.

\subsubsection{Review Heuristics}
After above, we need to brief the evaluators about the heuristics they are going to assess. And for setting up our own heuristics, Jakob Nielson mentioned 10 general principles for interaction design (Jakob Nielson, 1995) which can be regarded as the basic rules to follow when design the heuristics.

% Figure 2 Source:http://planbozchi24.wordpress.com/2013/09/21/heuristic-evaluation-of-rediffmail-an-existing-mail-client/

\begin{description}
  \item[Visibility of System Status]
    The system should always inform the users what it is undergoing.
  \item[Match between system and the real world]
    The system should provide the function of choosing national languages of users and show them in a familiar order for the users.
  \item[User control and freedom]
    Support undo, redo and quick control of the system.
  \item[Consistency and standards]
    Follow platform conventions.
  \item[Error prevention]
    Like the function that provide the users a confirmation option before they commit the action.
  \item[Recognition rather than recall]
    Minimize the users’ memory by making all the options and actions clear and visible.
  \item[Flexibility and efficiency of use]
    Allow the expert users to accelerate the process.
  \item[Aesthetic and minimalist design]
    Eliminate the irrelevant or unnecessary information in the interface or information box.
  \item[Help users recognize, diagnose, and recover from errors]
    Explain the error clearly and correctly when it occurs.
  \item[Help and documentation]
    Provide help or documentation for the better of use.
\end{description}

\subsubsection{Conduct the Evaluation}
To conduct the evaluation, the evaluators can work in groups or individually to review the interface and record the problems or findouts (Nicky Danino, 2001).

\subsubsection{Analyze the Results}
Once the evaluators have done their work, we should collect the data records from them and combine all the information with reducing the duplicated ones and address to the conclusion. And remember the golden rule “\emph{There is no such thing as a ‘user error}’” (Nicky Danino, 2001).

%
% Continue transfer here.
%
%


\subsection{Social Evaluation}
Social evaluation (SE) is a very important element within an effective evaluation system. One of the most important components of SE is to maintain the safety of work systems in the workplace. Safety of work systems is a function of many factors which affect the system, and these factors affect the safety of work systems simultaneously. For this reason, measuring work system safety needs a holistic approach. In this study, the work safety issue is studied through the analytic hierarchy process (AHP) approach which allows both multi-criteria and simultaneous evaluation. 

The AHP introduced by Saaty (1977) is a method addresses how to determine the relative importance of a set of activities in a multi-criteria decision problem. AHP aims to settle the conflict between practical demand and scientific decision making, and it also make it possible to find a way to blend process qualitative analysis and quantitative analysis (Mu, 1997). Decisions made using the AHP occur in two sequential phases: hierarchy design, which involves decomposing the decision problem into a hierarchy of interrelated decision elements (i.e., goal, and evaluation criteria); and hierarchy evaluation, which involves eliciting weights of the criteria and synthesizing these weights and preferences to determine alternative priorities (Sanjay and Ramachandran, 2006). One of the main advantages of the AHP method is the simple structure. Moreover, the use of AHP does not involve cumbersome mathematics, thus it is easy to understand and it can effectively handle both qualitative and quantitative data (Cengiz et al., 2003).

The AHP is one of the extensively used MCDM methods (Bagranoff, 1989, Arbel and Orgler, 1990 and Moutinho, 1993). It has been successfully used in maintenance policy selection (Arunraj and Maiti, 2010), environmental decision-making (Patrick and Laurence, 1999 and Chiang and Lai, 2002), resources planning (Willett and Sharda, 1991), and conflict management (Saaty, 1990a and Kang et al., 2007). AHP has also been employed for the risk assessment (Mustafa and Al-Bahar, 1991, Gaudenzi and Borghesi, 2006 and Zayed et al., 2008).
Another limitation faced in SE is the inability to measure the variables exactly and objectively. Generally, the factors affecting work system safety have non-physical structures. Therefore, the real problem can be represented in a better way by using fuzzy numbers instead of numbers to evaluate these factors. In this study, a fuzzy AHP approach is proposed to determine the level of faulty behavior risk (FBR) in work systems. In the application, factors causing faulty behavior are weighted with triangular fuzzy numbers in pairwise comparisons. These factors are evaluated based on the work system by using these weights and fuzzy linguistic variables. As a result of this evaluation FBR levels of work systems are determined and different studies are planned for work systems according to the FBR levels. In this way, faulty behavior is prevented before occurrence and work system safety is improved.

\section{Discussion}
\section{Summary}

% Bibliography
\cite{*}
\bibliographystyle{plainnat}
\bibliography{tddc34-report-stockholm-quickomat}
\end{document}